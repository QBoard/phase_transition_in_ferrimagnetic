\documentclass[aps,pra,twocolumn,amsmath,amssymb,nofootinbib,showpacs,superscriptaddress]{revtex4-1}
\usepackage[english]{babel}
\usepackage{latexsym}
\usepackage{graphics}
\usepackage{graphicx}
\usepackage{epsfig}
\usepackage{amsmath}
\usepackage{amssymb}
\usepackage{amsthm}
\usepackage{dcolumn}
\usepackage{float}
\usepackage{hyperref}
\usepackage{epstopdf}
\usepackage{cleveref}
\usepackage[svgnames]{xcolor}
\usepackage{enumerate}
\usepackage{braket}
\hypersetup{hidelinks,colorlinks=true,allcolors=DarkBlue}
\newcommand{\com}[1]{{\color{Red} #1}}

\renewcommand{\sb}{\mathbf{s}}
\newcommand{\sbx}{\mathbf{s_x}}
\newcommand{\sby}{\mathbf{s_y}}
\newcommand{\xb}{\mathbf{x}}
\newcommand{\xeb}{\mathbf{x}_{\mathrm{ext}}}
\newcommand{\xd}{\mathbf{x}_{\mathrm{dec}}}
\newcommand{\yb}{\mathbf{y}}
\newcommand{\yeb}{\mathbf{y}_{\mathrm{ext}}}
\newcommand{\eb}{\mathbf{e}}
\newcommand{\Hb}{\mathbf{H}}
\newcommand{\hb}{h_\mathrm{b}}
\newcommand{\rb}{\mathbf{r}}
\newcommand{\zb}{\mathbf{z}}
\newcommand{\epse}{\epsilon_\mathrm{est}}
\newcommand{\ed}{\eb_\mathrm{dec} }

\newcommand{\qe}{q_\mathrm{est}}

\newcommand{\Mb}{\mathbf{M}}
\newcommand{\LLR}{\mathrm{LLR}}
\newcommand{\Sm}{\mathcal{S}}
\newcommand{\Pm}{\mathcal{P}}
\newcommand{\Pmu}{\mathcal{P}_\mathrm{unt}}
\newcommand{\Nm}{\mathcal{N}}
\newcommand{\Mm}{\mathcal{M}}
\newcommand{\Am}{\mathcal{A}}
\newcommand{\fst}{f_{\mathrm{start}}}

\theoremstyle{remark}
\newtheorem{remark}{Remark}

\begin{document}

    \preprint{APS/123-QED}

    \title{Learning phase transitions in amorphous ferrimagnets}

    \author{N. Koritsky}
    \affiliation{Russian Quantum Center, Skolkovo, Moscow 143025, Russia}
    \affiliation{Moscow Institute of Physics and Technology, Dolgoprudny 141700, Russia}

    \author{S. Solov'yov}
    \affiliation{Moscow Institute of Physics and Technology, Dolgoprudny 141700, Russia}

    \author{A.K. Fedorov}
    \affiliation{Russian Quantum Center, Skolkovo, Moscow 143025, Russia}
    \affiliation{Moscow Institute of Physics and Technology, Dolgoprudny 141700, Russia}

    \author{A.K. Zvezdin}
    \affiliation{Russian Quantum Center, Skolkovo, Moscow 143025, Russia}
    \affiliation{Moscow Institute of Physics and Technology, Dolgoprudny 141700, Russia}
    \affiliation{Prokhorov General Physics Institute of the Russian Academy of Sciences, Moscow  119991, Russia}

    \date{\today}
    \begin{abstract}
        We identify phase transitions in weak amorphous ferromagnets using machine learning technique, which is based on the learning with confusion.
        We test the suggested method on the basis on the analytic model for the dynamics of domain walls in weak ferromagnets and obtain that corresponding universal $W$-shape.
        Then we discuss the application of such a method for the case of amorphous ferrimagnets, where parameters of the model may fluctuate.
        Special attention is paid to the case of GdFeCo, which is a metallic ferrimagnet with compensation point that is one of the most promising materials in ultrafast magnetism.
    \end{abstract}

    \maketitle



    \bibliography{biblio}
    \bibliographystyle{plain}
\end{document}
